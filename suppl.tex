\documentclass[11pt]{amsart} 
\usepackage{amsmath, amssymb, amsthm}
\usepackage[marginratio = 1:1,
     height = 601pt, 
     width = 435pt,
     tmargin = 95pt]{geometry}
\usepackage{listings}
 \lstset{language=R, 
     showspaces=false,
     showtabs=false,
     showstringspaces=false,
     basicstyle=\ttfamily
     \small,
     backgroundcolor=\color{mygray},
     frame=single,
     framerule=0.0pt}
\usepackage{graphicx}
\usepackage{xcolor}
\newcommand{\collineB}{0,0.3,0.8} 
\definecolor{mycolB}{rgb}{\collineB}
\definecolor{mygray}{rgb}{1,1,1}
\usepackage[colorlinks, 
     linkcolor=mycolB, 
     citecolor=blue,
     urlcolor=magenta]{hyperref}
\usepackage{verbatim}
% $$$:
%\renewcommand{\rmdefault}{ptm} % times
%\renewcommand*\ttdefault{txtt} % typewriter is txtt
%\renewcommand*\sfdefault{phv}  % helvetica
%\usepackage[subscriptcorrection]{mtpro}
%\usepackage[mtphrb]{mtpams}
%%\usepackage[mtpcal, mtpfrak]{mtpb}
\usepackage[scaled=1.1]{rsfso}
%
%\DeclareMathAlphabet{\txcal}{U}{tx-cal}{m}{n}
%\DeclareFontEncoding{FMS}{}{}
%  \DeclareFontSubstitution{FMS}{futm}{m}{n}
%\DeclareMathAlphabet{\foucal}{FMS}{futm}{m}{n}
%
\newcommand{\wP}{P^\ast}
\newcommand{\wE}{E^\ast}
\newcommand{\wZ}{Z^\ast}
\newcommand{\wt}{\theta^\ast}
\newcommand{\wk}{\psi^\ast}
\newcommand{\whp}{\widehat \pi}
\newcommand{\wAe}{A_e^\ast}
\newcommand{\wAp}{A_p^\ast}
\newcommand{\wBe}{B_e^\ast}
\newcommand{\wBp}{B_p^\ast}
%
\newtheorem{theorem}{Theorem}
\newtheorem{lemma}{Lemma}
\theoremstyle{definition}
\newtheorem{definition}[theorem]{Definition}
\newtheorem{example}[theorem]{Example}
\newtheorem{xca}[theorem]{Exercise}
\theoremstyle{remark}
\newtheorem{remark}[theorem]{Remark}

\setcounter{tocdepth}{3} % to get subsubsections in toc
\let\oldtocsection=\tocsection
\let\oldtocsubsection=\tocsubsection
\let\oldtocsubsubsection=\tocsubsubsection
\renewcommand{\tocsection}[2]{\hspace{0em}\oldtocsection{#1}{#2}}
\renewcommand{\tocsubsection}[2]{\hspace{.9em}\oldtocsubsection{#1}{#2}}
\renewcommand{\tocsubsubsection}[2]{\hspace{.5em}\oldtocsubsubsection{#1}{#2}}

%\makeatletter
%\def\@tocline#1#2#3#4#5#6#7{\relax
%  \ifnum #1>\c@tocdepth % then omit
%  \else
%    \par \addpenalty\@secpenalty\addvspace{#2}%
%    \begingroup \hyphenpenalty\@M
%    \@ifempty{#4}{%
%      \@tempdima\csname r@tocindent\psimber#1\endcsname\relax
%    }{%
%      \@tempdima#4\relax
%    }%
%    \parindent\z@ \leftskip#3\relax \advance\leftskip\@tempdima\relax
%    \rightskip\@pnumwidth plus4em \parfillskip-\@pnumwidth
%    #5\leavevmode\hskip-\@tempdima
%      \ifcase #1
%       \or\or \hskip .5em \or \hskip 2em \else \hskip 3em \fi%
%      #6\nobreak\relax
%    \dotfill\hbox to\@pnumwidth{\@tocpagenum{#7}}\par
%    \nobreak
%    \endgroup
%  \fi}
%\makeatother

\makeatletter
\def\@fnsymbol#1{%
  \ensuremath{%
    \ifcase#1% 0
    \or % 1
      \dagger%   
    \or % 2
      *
    \or % 3  
      \ddagger
    \or % 4   
      \mathsection
    \or % 5
      \mathparagraph
    \else % >= 6
      \@ctrerr  
    \fi
  }%   
}   
\makeatother

\renewcommand{\thefootnote}{\fnsymbol{footnote}}

\begin{document}
\title[R. A. Rosales, R. D. Drummond, R. Valieris, I. T. da Silva]{}
\noindent{\parbox{\linewidth}{\footnotesize %
    {\footnotesize\textsl{Submitted to Bioinformatics }}
    {\footnotesize\textrm{ 29/02/16 - v 0.5}}
  }
 }\\[1em]
 \begin{center}
   {\Large\bf Supplementary material to:\\[0.4em]
   An empirical Bayesian approach\\[0.3em] 
   to Mutational Signature Discovery\\[2em]}
   {\large 
     Rafael A. Rosales\footnote{R. A. Rosales and R. D. Drummond are
       both to be considered as First Author}, 
     Rodrigo D. Drummond$^\dagger$,
     Renan Valieris and 
     Israel T. da Silva\footnote{Corresponding
       author}$^,$\footnote{Partially supported by FAPESP grant */*}}
 \end{center}


\maketitle

%  {Rafael A. Rosales}\\
%  {\footnotesize\it Deparatmento de Computa\c{c}\~ao e Matem\'atica, 
%   Universidade de S\~ao Paulo}\\ 
%  {\footnotesize\it Av. Bandeirantes, 3900, Ribeir\~ao Preto, 
%  14049-901, SP, Brazil}\\{\footnotesize\verb+rrosales@usp.br+}\\[1em]
%%
%  {Rodrigo Drummond, Renan Valieris}\\
%  {\footnotesize\it Laboratory of Bioinformatics and Computational 
%  Biology, CIPE/A.C. Camargo Cancer Center\\ S\~ao Paulo 
%  01509-010,  Brazil}\\
%  {\footnotesize\verb+rddrummond@gmail.com+,    
%     \verb+rvalieris@gmail.com+}\\[1em] 
%%
%  {Israel T. da Silva}\\
%  {\footnotesize\it Laboratory of Bioinformatics and Computational 
%   Biology, CIPE/A.C. Camargo Cancer Center\\ S\~ao Paulo 
%   01509-010, Brazil}\\
%  {\footnotesize\it and}\\
%  {\footnotesize\it Laboratory of Molecular Immunology, 
%  The Rockefeller University}\\
%  {\footnotesize\it 1230 York Avenue, New York, NY 10065}\\
%  {\footnotesize\verb+itojal@gmail.com+}\\[1.5em]
%\end{center} 

\tableofcontents 

\section{Notation and preliminary Results}
Let $M$ be matrix of mutation counts of dimension $K\times G$ and let
$m$ denote a particular sample for $M$. For the factorization $M=PE$,
the observations $(M)_{ij}$ are independent and Poisson distributed
random variables with rates $(PE)_{ij}(W)_{ij}$, with $W$ as the
$K\times G$ opportunity matrix. The factors $P$, $E$, identified as
the model parameters are denoted as $\theta$. The likelihood for this
model, i.e. the function $\mathcal L: \Theta \to \mathbb R$ defined by
the map $\theta \mapsto p(M=m|\theta)$, is given by
\begin{equation}
  \label{eqn:PoisLik}
   \mathcal L(\theta; m) 
   =
    \prod_{i=1}^K \prod_{j=1}^G e^{-w_{ij}\sum_{n=1}^{N}p_{in}e_{nj}}
    \Big(w_{ij}\sum_{n=1}^{N}p_{in}e_{nj}\Big)^{m_{ij}}
    \frac{1}{m_{ij}!}.\tag{$s_1$}
\end{equation}

Posterior inferences about the factors $P$ and $E$ and the
factorization rank $N$ require the joint distribution for the
observations $M$ and the latent variables $Z$, also known as the
complete data likelihood function when interpreted as a function of
$\theta$ for given $Z=z$ and $M=m$. Following the condition
(1) in the main text and the independence between
the components of $Z$ this distribution equals
\begin{equation}
   \label{eqn:jointdata}
 \begin{aligned}
    p(Z = z, M =&\ m \mid P, E) \notag\\
  &= 
    \prod_{n=1}^N\prod_{i=1}^K\prod_{j=1}^G p\Big(Z_{inj} = z_{inj},
    M_{ij} = m_{ij}, \mathbf{1}_{\big\{M_{ij} = \sum_{n=1}^N
      Z_{inj}\big\}} \Big| P, E\Big)  \notag\\ 
  &=
    \prod_{n=1}^N\prod_{i=1}^K\prod_{j=1}^G e^{-p_{in}e_{nj}w_{ij}}
    (p_{in}e_{nj}w_{ij})^{z_{inj}} \frac{1}{z_{inj}!}
    \mathbf{1}_{\big\{m_{ij} = \sum_{n=1}^N z_{inj}\big\}}.
 \end{aligned}
 \tag{$s_2$}
\end{equation}
The symbol $\mathbf{1}$ denotes the indicator function for the 
event $\big\{M_{ij} = \sum_n Z_{inj}\big\}$, with value 1 if $M_{ij} =  
\sum_n  Z_{inj}$ and 0 otherwise. Marginalization of
(\ref{eqn:jointdata}) with respect to $Z$ gives  (\ref{eqn:PoisLik}).


\section{Gibbs sampler}
\label{sec:Gibbs}
The construction of the Gibbs sampler relies on the determination of
the full conditional distributions for all unknowns in the
hierarchical model, that is, the distribution of each component of
$(Z, \theta, \psi)$ conditioned on all other components of this vector,
the data $M$ and the hyperprior parameters $\eta$,
\[
   Z \sim \pi(Z|\theta, \psi, M, \eta), \quad
   \theta \sim \pi(\theta|Z, \psi, M, \eta) \quad
  \text{and}\quad
  \psi \sim \pi(\psi|Z, \theta, M, \eta).
\]
These distributions are obtained by straightforward computations
following the likelihood in (\ref{eqn:PoisLik}) and the hierarchical
model defined in the main text. The full conditional for $Z$ is
derived in Lemma~\ref{lem:Full_for_Z}. The full conditional for
$\theta$ follows by observing that $\pi(\theta|Z, \psi, M, \eta) =
p(P|\psi)p(E|\psi)$ \textcolor{red}{FINISH...}


The full conditional of $p_{in}$, for any $1 \leq i\leq K$ and $1\leq
n\leq  N$ is the following Gamma density,
\begin{equation}
  \label{eqn:Full_for_P}
  p_{in} 
       \sim 
     \text{Gamma}\Big(p_{in}\,\Big|\, \alpha_{in}^p + 1 +
     \sum_{j=1}^G z_{inj}, \beta_{in}^p + \delta_p +
     \sum_{j=1}^G e_{nj}w_{ij}\Big).\tag{$s_2$}
\end{equation}
Similarly, the full conditional for $e_{nj}$, for any $1\leq n\leq 
N$, $1\leq j\leq G$, follows the Gamma density  
\begin{equation}
  \label{eqn:Full_for_E}
  e_{nj} 
     \sim 
   \text{Gamma}\Big(e_{nj}\,\Big|\, \alpha_{nj}^e + 1 +
   \sum_{i=1}^K z_{inj}, \beta_{nj}^e + \delta_e +
   \sum_{i=1}^K p_{in}w_{ij}\Big).\tag{$s_3$}
\end{equation}


As part of the  empirical Bayesian approach, we also consider the
following full conditional distributions for the hyperparameters
$\psi$. Up to proportionality, each entry of the $B_p$ matrix,
$\beta_{in}^p$,  has the full conditional density
\begin{equation}
 \label{eqn:Full_for_Bp}
 \beta_{in}^p
   \propto
      (\beta_{in}^p + \delta_p)^{\alpha_{in}^p + 1}(\beta_{in}^p)^{a_p
      - 1} \exp\Big(-(p_{in}+b_p)\beta_{in}^p\Big),\tag{$s_4$}
\end{equation}
for $\beta_{in}^p > 0$. When $\delta_p = 0$, this  corresponds  to a
Gamma density with shape $\alpha_{in}^p + 1 + a_p$ and rate $p_{in} +
b_p$.  Likewise, for the $B_e$ matrix, the full conditional has density
\begin{equation}
 \label{eqn:Full_for_Be}
 \beta_{nj}^e
   \propto
      (\beta_{nj}^e + \delta_e)^{\alpha_{nj}^e + 1}(\beta_{nj}^e)^{a_e
      - 1} \exp\Big(-(e_{nj}+b_e)\beta_{nj}^e\Big),\tag{$s_5$}
\end{equation}
for $\beta_{nj}^e > 0$. Also, if $\delta_e = 0$, this density is a
Gamma with shape $\alpha_{nj}^e + 1+ a_e$ and rate $e_{nj}+b_e$. It
should be observed that the densities for the cases $\delta_p > 0$ and
$\delta_e > 0$ are not from a standard family of known densities. In
this case, samples from the associated distributions may be generated
by using Metropolis-Hastings steps. Further details concerning their
implementation are described in Section~\ref{sec:MHsteps} here.



The exponential prior for the elements of $A_p$ leads, up to
proportionality, to the following the full conditional density
for\footnote{\textcolor{red}{RD: I know that we already went though
    this, but  is it safe to take the constant $\lambda_p(\beta_{in}^p
    + \delta_p)$ away from the density for $\alpha_{in}^p$ --because
    it  cancels when one divides by the integral that will define the  normalising constant?}}  
$\alpha_{ij}^p$,
\begin{equation}
 \label{eqn:Full_for_Ap}
  \alpha_{in}^p 
  \sim 
  \lambda_p\frac{(\beta_{in}^p + \delta_p)}{\Gamma(\alpha_{in}^p + 1)} 
   \Big[(\beta_{in}^p + \delta_p)p_{in}
   e^{-\lambda_p}\Big]^{\alpha_{in}^p}, \tag{$s_6$}
   \quad  \alpha_{in}^p > 0.
\end{equation}
Similarly, up to proportionality, for the elements of $A_e$ we
have\footnote{\textcolor{red}{RD: same here, should we take away the constant
  $\lambda_e(\beta_{nj}^e + \delta_e)$ in the density for
  $\alpha_{nj}^e$?}}
\begin{equation}
 \label{eqn:Full_for_Ae}
  \alpha_{nj}^e 
  \sim
  \lambda_e\frac{(\beta_{nj}^e + \delta_e)}{\Gamma(\alpha_{nj}^e + 1)} 
   \Big[(\beta_{nj}^e + \delta_e)e_{nj}
   e^{-\lambda_e}\Big]^{\alpha_{nj}^e}, 
   \quad  \alpha_{nj}^e > 0. \tag{$s_7$}
\end{equation}
The full conditional distributions for the hyperparameters $A_p$ and
$A_e$ are not from a standard family. Samples from these distributions
are obtained in this case also  by considering Metropolis-Hastings
steps. These are presented in Section~\ref{sec:MHsteps} here.


The full conditional distribution for the latent variables $Z$ also
has to be determined. In particular, for the model considered
throughout, the distribution of $Z$ is determined by a multinomial
distributions. This result is precisely stated by the following
Lemma.

\begin{lemma}\label{lem:Full_for_Z} Conditionally on  $M= m$, the full
  conditional distribution for the latent variables $Z$ is the
  following  product of multinomial laws 
\[
   p(Z = z\,|\, \theta, \psi, M=m, \eta)
   = 
   \prod_{i=1}^K\prod_{j=1}^G {m_{ij} \choose z_{i1j}, \ldots,
     z_{iNj}} 
   \prod_{n=1}^N \phi_{inj}^{z_{inj}},
\]
with $\phi_{inj} = p_{in} e_{nj}/\sum_{r=1}^N p_{ir}e_{rj}$.
\end{lemma}

\begin{proof} 
Conditionally on $M = m$, $P$ and $E$, the distribution of $Z$ is
independent of $\psi$ and $\eta$, $p(Z\,|\, M$, $P$, $E$, $\psi$, $\eta)
= p(Z\,|\, M$, $P$, $E)$, because by definition $Z$ is a set of
Poisson random variables with rates determined by $P$ and $E$. The
required full conditional is determined by the ratio $p(Z\,|\,M, P, E)
= p(Z,  M\,|\, P,E)/p(M\,|\,P, E)$, namely by considering the ratio 
of (\ref{eqn:jointdata}) by (\ref{eqn:PoisLik}). Taking logarithms
leads to
\begin{align*}
    \ln p(Z|M, P, E) 
  =&
    \ln p(Z, M|P, E) - \mathcal L(\theta; m) \\
  =&
    \sum_{i=1}^K \sum_{j=1}^G \sum_{n=1}^N z_{inj}
     \ln(p_{in}e_{nj}w_{ij}) - 
     p_{in}e_{nj}w_{ij} - \ln(z_{inj}!) +\ln \mathbf{1}_{\big\{M_{ij} =
     \sum_{n=1}^N Z_{inj}\big\}} \\ 
    & -\sum_{i=1}^K \sum_{j=1}^G m_{ij}\ln \sum_{u=1}^N
      p_{iu}e_{uj}w_{ij} - 
      \sum_{u=1}^N p_{iu}e_{uj}w_{ij} + \ln m_{ij}!.
\end{align*}
Direct simplifications obtained by  setting $m_{ij}$ equal to
$\sum_{n=1}^N  z_{inj}$ lead to  
\begin{align*}
    \ln p(Z|M, P, E) 
  =&
    \sum_{i=1}^K \sum_{j=1}^G\Big\{
       \sum_{n=1}^N \Big(
           z_{inj} \ln\frac{p_{in}e_{nj}}{\sum_{u=1}^N
           p_{iu}e_{uj}} - \ln z_{inj}!
       \Big) + 
       \ln \mathbf{1}_{\big\{M_{ij} = \sum_{n=1}^N Z_{inj}
     \big\}} \\
     &+ \ln \big(\sum_{u=1}^N z_{iuj}\big)!
   \Big\}.
\end{align*}
Considering exponentiation to revert to the original scale shows that
the full conditional for $Z$ is fact a product of multinomial
distributions as required,
\begin{align}
       p(Z = z\mid M = m, P, E) 
     &= 
       \prod_{i=1}^K \prod_{j=1}^G
       p(Z_{i1j} = z_{i1j}, \ldots, Z_{iNj} =
       z_{iNj}\mid m_{ij}, P, E) \notag\\ 
     &=
       \prod_{i=1}^K \prod_{j=1}^G
       \frac{m_{ij}!}{z_{i1j}!\cdots z_{iNj}!}
        \phi_{i1j}^{z_{i1j}} \cdots
       \phi_{iNj}^{z_{iNj}}. \notag\qedhere
\end{align}
\end{proof}


\section{MCMC implementation details}\label{sec:MHsteps}
The full conditional distributions for the entries in both $A_p$ and
$A_e$ do not have a standard form. These are explicitly shown in
(\ref{eqn:Full_for_Ap}) and (\ref{eqn:Full_for_Ae}). Draws from these
distributions, necessary to define the Gibbs sampler,  are obtained by
using Metropolis-Hastings steps.  Let $x > 0$ be the current value for
for any given entry of $A_e$ or $A_p$. A new candidate value $y$ for
this variable is generated from a Gamma proposal density, $g(\cdot|
x)$ with shape $(x/\sigma)^2$ and rate $x/\sigma^2$. The mean of this
proposal is set equal to $x$ and its variance to $\sigma^2$. The 
results  described in the main text where obtained by
\textcolor{red}{choosing 
  $\sigma^2 = *$}\footnote{\textcolor{red}{RD: please put the value
    here!}}. Let $\alpha$ be the ratio,
\[
  \alpha 
 =
  \frac{p(y)}{p(x)} \frac{g(x|y)}{g(y|x)}
\]
with $p(x)$ as any one of the densities in (\ref{eqn:Full_for_Ap}) and 
(\ref{eqn:Full_for_Ae}). The value $y$ is accepted as a sample
with probability $\rho = \min\{1, \alpha\}$, that is, if $x'$ denotes
the new sampled value, then
\[
   x'
    =
  \begin{cases}
    x, & \text{if } U > \rho,\\
    y, & \text{if } U \leq \rho.
  \end{cases}
\]
for $U$ an uniformly distributed random variable on $[0, 1]$.


The samples for the entries in $B_p$ and $B_e$ are also produced by
Metropolis-Hastings steps whenever $\delta_p > ...$
\textcolor{red}{Finish  this!}.

\section{MCMC EM}
\subsection{Maximisation step}
This section describes the maximization step in the EM algorithm
necessary to derive the estimator $\hat\eta$, namely
\begin{equation}
   \label{eqn:etaMAX}
    \underset{\eta\,\in\,\Lambda}{\text{arg max}}\,
    \frac{1}{R}\sum_{r=1}^R \ln p\big(M, Z^{(r)}, \theta^{(r)},
    \psi^{(r)}\,|\, \eta\big). \tag{$s_9$}
\end{equation}
The solution to (\ref{eqn:etaMAX}) is used to define the sequence
$\eta^{(k)}$, $k \geq 1$. The components of the latter are denoted
throughout as $\lambda_a^{(k)}$, $a_e^{(k)}$ and so on.  Let
$\ell(\eta) = \ln p(M, Z^{(r)}, \theta^{(r)}, \psi^{(r)}\,|\,
\eta)$. The maximization of $\frac{1}{R}\sum_{r}\ell(\eta)$ with
respect to $\eta$ is made by solving
$\nabla\frac{1}{R}\sum_{r}\ell(\eta) = 0$ for $\eta$.  To start,
observe that $\ell(\eta)$ admits the form
\begin{equation}
 \label{eqn:ell}
 \begin{aligned}
  \ell(\eta)
  =&
  \ln p(M, Z^{(r)} | \theta^{(r)})  + \ln p(P^{(r)}\,|\, A_p^{(r)},
  B_p^{(r)}) + \ln p(E^{(r)}\,|\, A_e^{(r)},  B_e^{(r)})  \\
  &+
  \ln p(A_p^{(r)}\,|\, \eta) + \ln p(B_p^{(r)}\,|\, \eta) +
  \ln p(A_e^{(r)}\,|\, \eta) + \ln p(B_e^{(r)}\,|\, \eta).
 \end{aligned}
 \tag{$s_{10}$}
\end{equation}
The first three terms of $\ell(\eta)$ are independent of
$\eta$. Further, each of the remaining terms can be optimized
separately because each one depends on a different component of
$\eta$. Following the exponential hyperprior for $A_p^{(r)}$ we have
\[
   \frac{1}{R}\sum_{r=1}^R \ln  p(A_p^{(r)}\,|\ \lambda_p)
  %=
  % \frac{1}{R}\sum_{r=1}^R \sum_{i=1}^K\sum_{n=1}^N
  % \big(\ln(\lambda_p) - \lambda_p (A_p^{(r)})_{in}\big)
  = 
  \frac{1}{R}KN\ln(\lambda_p) - \frac{1}{R}\lambda_p\sum_{r=1}^R
  \sum_{i=1}^K\sum_{n=1}^N (A_p^{(r)})_{in}.
\]
Hence, solving for $\lambda_p$ in $\partial (\frac{1}{R}\sum_r \ln
p(A_p^{(r)}\,|\ \lambda_p))/\partial\lambda_p = 0$ gives
\[
  \lambda_p^{(k)} = \frac{RKN}{\sum_{r=1}^R \sum_{i=1}^K 
    \sum_{n=1}^N (A_p^{(r)})_{in}}.
\]
Likewise, by considering the third of the remaining terms in
(\ref{eqn:ell}), namely $\frac{1}{R} \sum_r 
p(A_e\,|\, \lambda_e)$, gives
\[
  \lambda_e^{(k)} = \frac{RNG}{\sum_{r=1}^R \sum_{n=1}^N
    \sum_{j=1}^G (A_e^{(r)})_{nj}}.
\]
The maximization of the second and the fourth terms in (\ref{eqn:ell})
must be handled numerically. This situation is analogous to the
maximum likelihood estimation of the scale and the rate parameters of
a Gamma density, which has no closed form solution, see \cite{CW}. For
instance, because of a Gamma hyperprior for $B_p$ with parameters
$a_p$ and $b_p$, for the second term we have that
\[
   \frac{1}{R}\sum_{r=1}^R \ln p(B_p^{(r)}\,|\, \eta)
  =
   \frac{1}{R}\sum_{r=1}^R \sum_{i}\sum_{n} a_p\ln b_p  
    - \ln\Gamma(a_p) + 
  (a_p-1)\ln\big[(B_p^{(r)})_{in} - b_p(B_p^{(r)})_{in}\big].
\]
Solving for $b_p$ in $\partial(\frac{1}{R}\sum_r \ln p(B_p^{(r)}\,|\,
\eta))/\partial b_p = 0$  yields\footnote{\textcolor{red}{RD, RR: have
    to check this last equation!}}
\[
   b_p =\frac{1}{R} \frac{\sum_r\sum_i\sum_n (B_p^{(r)})_{in}}{a_p}.
\] 
\textcolor{red}{Rodrigo: still to write} the numerical implementation
of how $b_p$ (and hence also $b_e$) are obtained. Maybe you are using
\cite{J}, in any case please explain.


\subsection{Convergence}
This section justifies the convergence of $\hat\pi(\theta|M,
\hat\eta)$ as defined by (3) towards $\pi(\theta|M, \eta)$. The
arguments follow closely those in \cite{C01}, but they are adapted to
the special case of the Metropolis-within-Gibbs sampler considered
throughout. 

\begin{lemma}\label{lem:technical} Suppose that the following
  conditions hold:
\begin{itemize}
 \item[(i)] $\hat\eta^{(k)} \to \eta$ as $k \to \infty$ almost
   surely with respect to $m$; 
 \item[(ii)] $h_\eta(\eta')$ defined as
\[
  h_{\eta}(\eta') 
   = 
 \int p(\theta|Z, \psi, M, \eta')p(Z, \psi|M, \eta)\ dZd\psi
\]
is continuous in both $\eta$ and $\eta'$;
\item[(iii)] $\widehat h_\eta(\eta')$, defined as
\begin{align*}
  \widehat h_\eta(\eta') 
  = 
 \frac{1}{R}\sum_{r=1}^R \pi(\theta|Z^{(r)}, \psi^{(r)}, M, \eta')
    \quad \text{with}\quad 
   \begin{aligned}
      &\psi^{(r)} \sim \pi(\psi|Z^{(r-1)}, \theta^{(r)}, M, \eta)\\
      &Z^{(r)} \sim \pi(Z| \psi^{(r)}, \theta^{(r)}, M, \eta) 
   \end{aligned}
\end{align*}
 is continuous in $\eta'$ and stochastically equicontinuous in $\eta$,
that is, for any given $\epsilon > 0$ there is $\delta>0$ such that
$|\eta_1 - \eta_2| < \delta$ implies $|\widehat h_{\eta_1}(\eta') -
\widehat h_{\eta_2}(\eta')|$\ \ for all $\eta_1, \eta_2$ except for a
$\pi$-measurable null set.
 \item[(iv)] the Metropolis-within-Gibbs sampler produces an ergodic Markov chain.
\end{itemize}
Then there exists a subsequence $(r_u)$ such that $\lim_{u\to\infty}
r_u  = \infty$, for which
\[
   \big|\widehat h_{\hat\eta_u}(\hat\eta_u) - h_\eta(\eta)\big| \to 0 
  \quad\text{as}\quad 
   u \to \infty
\]
almost surely with respect to the densities * and *.
\end{lemma}
\begin{proof} The existence of $(r_u)$ ensuring the required
convergence follows from Lemma~A.1 in \cite{C01}.  The rest of the
proof consists in a direct verification of the hypothesis (i)-(iv) for
the context of the current application.


The first assertion follows by the argument exposed in the main text, 
namely ... The second assertion follows by the continuity of the full
conditional densities for $\theta$ and $\psi|M, \eta$ with respect to
$\eta'$ and $\eta$ respectively. Likewise, (iii) follows by ... The
ergodicity of the Metropolis-within-Gibbs sampler considered
throughout follows from the Harris recurrence of the Markov chain
$(Z^{(r)}, \theta^{(r)}, \psi^{(r)})$, $r \geqslant 1$ , see Theorem
12 in \cite{RR}. Harris recurrence is established by observing that
\textcolor{red}{... finish this!}.
\end{proof}

\begin{theorem} Under the conditions in Lemma~\ref{lem:technical}, for
  each measurable set $T\subseteq\Theta$, as $R$ and $k \to \infty$,
\[
  \int_T \bigg|
       \frac{1}{R}\sum_{r=1}^R \pi\big(\theta|Z^{(r)}, \psi^{(r)},
       M, \hat\eta^{(k)}\big) - \pi(\theta|M, \eta)
    \bigg|\ d\theta \to 0
\]
almost surely with respect to the measures with densities $g$  and
$m$.
\end{theorem}
\begin{proof}
This Theorem is a immediate consequence of Lemma~\ref{lem:technical}
and Scheff\'e's Lemma, see \cite{C01} for further details.
\end{proof}


\section{Installing and running \texttt{signeR}}
\subsection{Installing \texttt{signeR}}
\texttt{signeR} is available as an R package and it can be installed
from R's prompt by typing
\begin{lstlisting}[]
  install.packages('signeR', dependencies=TRUE)
\end{lstlisting}
This should install also the following dependencies:
\textcolor{red}{....  Now comes the difficult bit: are we distributing
pre-compiled binaries at CRAN, or just source? Note that the first
option needs someone to mantain this over time. In the second case,
the installation relies upon the compilation of C++ code, and hence
depends on the existence of a properly working C compiler. Have to
explain on how to do this in a Mac OS X and Windows.}


\subsection{Running \texttt{signeR}}
Write: A step-by-step example about how \texttt{signeR} would be run
on a real data example, probably on the 21 breast cancer data.
Loading the package \verb+library(signeR)+ and then typing
\verb+eBayesNMF()+ with the following arguments. Explain the data
structure for the input matrices $M$, $W$.


Mention the actual dependencies \cite{Boo}, etc, and the also
R's NMF package.
\begin{lstlisting}[]
  # this is a comment
  Mut<-read.table('21_breast_cancers.mutations.txt',header=FALSE)
  Opp<-read.table('21_breast_cancers.opportunity.txt',header=FALSE)
  Outsig<-signeR(M=Mut, Opport=Opp)  
  Paths(Outsig$SignExp)
  SignPlot(Outsig$SignExp)
  SignBoxPlot(Outsig$SignExp)
  Classify(Outsig$SignExp, labels=c(rep(`GradeII',10),
     rep(`GradeIII',10), NA))
  DiffExp(Outsig$SignExp, labels=c(rep(`GradeII',10),
     rep(`GradeIII',11)))
\end{lstlisting}

%\bibliographystyle: natbib, achemnat, plainnat, abbrv, plain 
\bibliographystyle{alpha}
\bibliography{suppl}


\vspace{1cm}

{\footnotesize
\begin{tabular}{ll}
 \centering
  \parbox[c][4cm][t]{8cm}{
   {\sc Rafael A. Rosales}\\
   {\it 
   Departamento de Computa\c{c}\~ao e Matem\'atica\\
   Universidade de S\~ao Paulo\\
   Av. Bandeirantes, 3900, Ribeir\~ao Preto\\
   CEP 14049-901,SP Brazil\\}
   E-mail: \verb~rrosales@usp.br~
  }
&
  \parbox[c][4cm][t]{6.6cm}{
   {\sc Rodrigo D. Drummond\\
       Renan Valieris}\\
    {\it 
    Laboratory of Bioinformatics and Computational\\
    Biology, CIPE/A.C. Camargo Cancer Center\\ 
    S\~ao Paulo 01509-010, Brazil\\}
    E-mails: \parbox[t]{2.5cm}{%
      \verb~rddrummond@gmail.com~\\ 
      \verb~rvalieris@gmail.com~}
 }
\\[-1em]
  \parbox[c][4cm][t]{6.6cm}{
   {\sc Israel T. da Silva}\\
   {\it 
    Laboratory of Bioinformatics and Computational\\
    Biology, CIPE/A.C. Camargo Cancer Center\\ 
    S\~ao Paulo 01509-010, Brazil\\
    and\\
    Laboratory of Molecular Immunology\\
    The Rockefeller University\\
    1230 York Avenue, New York, NY 10065\\}
    E-mail: \verb~itojal@gmail.com~
 }
\end{tabular}
}

\end{document}
%%% Local Variables:
%%% mode: latex
%%% TeX-master: t
%%% End:
