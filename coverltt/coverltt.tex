\documentclass[11pt,a4paper]{letter} % 
\usepackage{graphicx} % Required for including pictures
\usepackage{microtype} % Improves typography
\usepackage[T1]{fontenc} % Required for accented characters
\usepackage{sabon}
\usepackage{wasysym}
\usepackage[
    colorlinks, linkcolor={blue}, citecolor={blue},
    urlcolor={blue}]{hyperref}  
\renewcommand*\ttdefault{txtt} 
\usepackage{marvosym}
\usepackage{xcolor}
\definecolor{ACCgray}{RGB}{130,130,130}


% Create a new command for the horizontal rule in the document which
% allows thickness specification
\makeatletter
\def\vhrulefill#1{\leavevmode\leaders\hrule\@height#1\hfill \kern\z@}
\makeatother

\usepackage[left=0.75in, right=0.75in, top=1in, bottom=0.3in]{geometry}
%\textwidth 6.75in
%\textheight 15.25in
%\oddsidemargin -.25in
%\evensidemargin -.25in
%\topmargin -2.5in
%\longindentation 0.50\textwidth
%\parindent 0.4in

%-------------------------------------------------------------------------
%	SENDER INFORMATION
%-------------------------------------------------------------------------

\def\Who{Israel Tojal da Silva} % Your name
\def\What{, PhD} % Your title
\def\Where{Computational Biology and \\ Bioinformatics Group} % Your department/institution
\def\Address{Rua Tagu\'a, 440} % Your address
\def\CityZip{S\~{a}o Paulo SP 01508-010} % Your city, zip code, country
\def\Emailits{\Email\  \texttt{itojal@cipe.accamargo.org.br}} % email
\def\TEL{\phone\  $+$55(11) 2189-5000 {\bf ext}: 2941} % Your phone number
\def\URL{\Mundus\  \texttt{http://www.accamargo.org.br}} % Your URL

%-------------------------------------------------------------------------
%	HEADER AND FROM ADDRESS STRUCTURE
%-------------------------------------------------------------------------
\address{
\includegraphics[width=0.8in]{logo} % Include the logo 
\hspace{5.8in} % Position of the institution logo, increase to move left, decrease to move right
\vskip -1.07in~\\ % Position of the text in relation to the institution
\Large\hspace{1.0in}\textsf{A.C. Camargo}\hfill ~\\[0.01in] % First line of institution name, adjust hspace if your logo is wide
\normalsize\hspace{1.0in}\textsf{Cancer Center}\hfill \normalsize % Second line of institution name, adjust hspace if your logo is wide
\makebox[0ex][r]{\bf \Who \What }\hspace{0.75in} % Print your name
~\\[-0.05in] % Reduce the whitespace above the horizontal rule
\hspace{1.in}\vhrulefill{1pt} \\ % Horizontal rule, adjust hspace if your logo is wide and \vhrulefill for the thickness of the rule
\hspace{9.7cm}\parbox[t]{2.85in}{ % Create a box for your details underneath the horizontal rule on the right
\footnotesize % Use a smaller font size for the details
% \Who \\ \em % Your name, all text after this will be italicized
\Where\\ % Your department
\Address, % Your address
\CityZip\\ % Your city and zip codeB
\TEL\\ % Your phone number
\Emailits\\ % Your email address
\URL % Your URL
}
\hspace{-1.0in} % Horizontal position of this block, increase to move left, decrease to move right
\vspace{-1in} % Move the letter content up for a more compact look
}

%-------------------------------------------------------------------------
%	TO ADDRESS STRUCTURE
%-------------------------------------------------------------------------

\def\opening#1{\thispagestyle{empty}
{\centering\fromaddress \vspace{0.6in} \\ % Print the header and from address here, add whitespace to move date down
\hspace*{\longindentation}\hspace*{\fill}\par} % Print today's date, remove \today to not display it
{\raggedright \toname \\ \toaddress \par} % Print the to name and address
\vspace{0.8in} % White space after the to address
\noindent #1 % Print the opening line
% Uncomment the 4 lines below to print a footnote with custom text
\def\thefootnote{}
\def\footnoterule{\hrule}
\footnotetext{\centering{\scriptsize {\textcolor{ACCgray}{Rua Tagu\'a, 440
 - Liberdade 01508-010 S\~ao Paulo, SP - Brazil $\star$ \phone\ $+$ 55 11
  21895000 $\star$ \FAX\ $+$ 55 11 21895017}}}}
\def\thefootnote{\arabic{footnote}}
}

%-------------------------------------------------------------------------
%	SIGNATURE STRUCTURE
%-------------------------------------------------------------------------

\signature{\Who \What} % The signature is a combination of name/title
\long\def\closing#1{
\vspace{0.4in} % Some whitespace after the letter content and before signa
\noindent % Stop paragraph indentation
\hspace*{\longindentation} % Move the signature right
\parbox{\indentedwidth}{\raggedright
#1 % Print the signature text
\vskip 0.65in % Whitespace between the signature text and your name
\fromsig}} % Print your name and title

%-------------------------------------------------------------------------

\begin{document}

%-------------------------------------------------------------------------
%	TO ADDRESS
%-------------------------------------------------------------------------

\begin{letter}
{}

%-------------------------------------------------------------------------
%	LETTER CONTENT
%-------------------------------------------------------------------------

\opening{Dear Editor,}

We are submitting a manuscript entitled \emph{signeR: An empirical Bayesian 
 approach to mutational signature discovery}, which we would like to
have considered for review by \textsc{Bioinformatics}. 

It is currently recognized that the accumulation of genomic
alterations is one of the major causes of malignant
transformation. Knowledge of the processes that trigger
the involved somatic mutations still remains elusive. These
processes have been identified with mutational signatures while
analysing high throughput data via non-negative matrix factorisation
(NMF) methods.  These efforts have advanced current knowledge about
the development of cancers and have recently received
substantial interest in the literature. Current methods used
for the identification of mutational signatures are however strongly
dependent upon initial conditions and address the determination of the 
underlying number of signatures by using ad-hoc heuristics.

In this manuscript we present a novel method for the statistical
estimation of mutational signatures based on an empirical Bayesian
treatment of the NMF model. Our method addresses the determination of
the number of signatures directly as a model selection problem. In
addition, we introduce two new concepts, of significant clinical
relevance for evaluating the mutational profile: the differential
exposure score and the posterior classification of genome samples. The
advantages brought by our approach are shown by the analysis of real
and synthetic data sets. Our approach is robust to initial conditions
and more accurate than competing alternatives which consider the same
NMF model. It also estimates the correct number of signatures even
when other methods fail.

%We have made our code available at
%\url{https://github.com/rvalieris/signeR}. A version submitted to 
%Bioconductor should be available shortly. Please also note that the 
%supplementary material includes a list of instructions about how to
%install and run this code. Last, to aid the reviewing process in case
%there is any trouble during the installation process, we also
%set up a Rstudio server at \url{http://143.107.223.173:8787/}. Login
%instructions are described Supplementary File for Review Only.


Our work is grounded on a solid statistical basis and we believe that
it is attractive to a broad spectrum of computational biology
scientists and those of the cancer genomics field and therefore hope
you will consider it for publication in \textsc{Bioinformatics}.

\hfill\today\\

Sincerely yours,\\
Rafael A. Rosales\\ 
Rodrigo Drummond\\ 
Renan Valieris\\ 
Emmanual Dias-Neto\\
Israel Tojal da Silva\\[1em]

%Should you decide to review the manuscript, we would like to ask that
%the authors of two recent alternative methods namely Ludmil
%B. Alexandrov and Andrej Fisher, be omitted as referees for reasons of
%direct competition.

%Finally, I wanted to mention that the work is highly competitive and
%we would appreciate a rapid decision.

%\closing{Sincerely yours,}

%-------------------------------------------------------------------------

\end{letter}
\end{document}

%%% Local Variables:
%%% mode: latex
%%% TeX-master: t
%%% End:
